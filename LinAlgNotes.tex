\documentclass{article}
\usepackage{amsmath}
\usepackage{amssymb}

\begin{document}

\section{Feb 15 notes}

Suppose that the observations are a Hermitian matrix. If they are not, then start from the Gram matrix and take a matrix sqrt of this matrix to be the observation matrix.

We would like to analyze $||\Sigma||_\text{op}$.

$$\Sigma = D^T M^T M D$$

Where $M$ is the Hermitian data matrix, and $D$ is an n-by-k matrix:\\ $\left(\begin{matrix} e_{d_1} & e_{d_2} & ... & e_{d_k} \end{matrix}\right)$, selecting k columns of $M$ using multiplication on the right.

We introduced $u$, which can be initialized to be the left singular vector of $MD$ for some well-performing $D$ ($\sigma = |uMD|_2$ large). This is slightly useful, because inner products with $u$ could be factored out to be computed using entries of $\Sigma$ only, and discard/never synthesize an $M$.

Set $A = D^T M^T u u^T M D$ (large and rank-one), and $E = \Sigma - A$. Analyzing $E$ for different permissible $D$ should give us a smaller recursive problem, compared to analyzing $\Sigma$ (shrinking the problem). However, we would like to produce a new upper-bound deterministically in polynomial time instead.

We found that $E$ is similar to:

$$(I - \frac{1}{\sigma^2} M D u u^T D^T M) M^2$$

We hope that $||E||_\text{op}$ is small compared to $||\Sigma||_\text{op}$. Note that with a change-of-basis using the diagonalization of $M$, then the matrix in parentheses is still a rank-$n-1$ projection matrix. It could have a zero for the dominant eigenvalue of $M^2$, and that is the best case. We want a large sine of the angular comparison of the dominant eigenspaces of the projection matrix on the left and of $M$, and this sine can be bounded using the Davis-Kahan theorem.

Let $P = I - \frac{1}{\sigma^2} MDu u^T D^T M$. For Davis-Kahan, we should come up with our own upper-bound on $||P - \frac{1}{\sigma^2} M^2||_\text{op}$. Davis-Kahan bounds the dissimilarity in the largest eigenspaces of $P$ and $\frac{1}{\sigma^2} M^2$, and if this dissimilarity obtains its maximum value, then the dominant eigenvalue would drop out of $PM^2$ and $||PM^2||_\text{op}$ would be driven by the second-largest eigenvalue of $M^2$ instead.

\section{March 1 notes}

We would like a large sine distance between the eigenvector spanned by the range of $\frac{1}{\sigma^2} MDuu^T D^T M$, and the dominant eigenvector of $M$. We expect: $$\sin \angle(\frac{1}{\sigma^2} MDuu^T D^T M, M) = \cos \angle (I-\frac{1}{\sigma^2} MDuu^T D^T M, M)$$

We don't need to use $\sin^2\theta + \cos^2\theta = 1$, because we are actually swapping whether the vector or some orthogonal vector is used as $\sin$ or as $\cos$.

To improve performance of Davis-Kahan, scale up the rank-one projection matrix so that it is $MDuu^T D^T M$. Apply Davis-Kahan:

$$
|\sin\theta|
\le
\frac{2||M(I-Duu^TD^T)M||_F}{\min(||MDuu^TD^TM||_\text{op}, \text{Gap}[M^2])}
$$

We need a lower bound on $\text{Gap}[M^2]$. If our gap evaluation is intentionally simplistic (take $\lambda_2 = \text{Tr }M^2 - \lambda_1$), then the gap estimate is nondecreasing with $\lambda_1$. If we choose a putative eigenvector to evaluate $\lambda_1$ for $\text{Gap}[M^2]$, and later we increase our objective function, then our gap using the best sparse PCA solution so far will still hold as a lower bound. Therefore, we use Berk et. al's stochastic best-observed lower bound solution to compute the gap.

\end{document}